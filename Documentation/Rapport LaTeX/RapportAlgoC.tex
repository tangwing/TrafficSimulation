
\documentclass[twoside]{EPURapport}
\usepackage{listings}

%\renewcommand{\lstlistlistingname}{Liste des codes}
%\renewcommand{\lstlistingname}{Code}

%\addextratables{%
%	\lstlistoflistings
%}

%\swapAuthorsAndSupervisors



\nolistoftables
\thedocument{Rapport de projet d'Algorithme et de Langage C}{Simulation du trafic urbain}{Simulation du trafic}

\grade{D�partement Informatique\\ 3\ieme{} ann�e\\ 2011 - 2012}

\authors{%
	\category{�tudiants}{%
		\name{Lei SHANG} \mail{lei.shang@etu.univ-tours.fr}
		\name{Yanxiao HU} \mail{yanxiao.hu@etu.univ-tours.fr}
	}
	\details{DI3 2011 - 2012}
}

\supervisors{%
	\category{Encadrants}{%
		\name{N�ron EMMANUEL} \mail{emmanuel.neron@univ-tours.fr}
	}
	\details{Universit� Fran�ois-Rabelais, Tours}
}

\abstracts{Le \textit{Mod�le de Conducteur Int�lligent}(IDM) est un mod�le qui repr�sente l'action des conducteurs, c'est-�-dire le mouvement raisonable des voitures dans un trafic urbain. Nous avons �tabli l'algorithme � partir de ce mod�le pour simuler visuellement sur l'�cran, une file de voiture sur la route avec 2 feux de circulation. L'application que nous avons r�alis� peut �tre fortement configurable � l'aide du ficher de configuration.}
{IDM, Simulation du trafic, Algorithme, OpenGL}
{The \textit{Intelligent Driver Model}(IDM) is a time-continuous car-following model for the simulation of freeway and urban traffic. According to this model, we have proposed an algorithm and then an application to simulate traffic on the screen. This application is developped with C programming language and can be highly custumized with the help of the configuration file.}
{IDM, Traffic simulation, Algorithm, OpenGL}


\begin{document}

\chapter{Introduction}
Ce projet est d�velopp� dans le cadre de la formation de l'algorithme et le langage C en troisi�me ann�e du d�partement informatique. Le but du projet est de pratiquer les connaissances que nous avons acquises durant les s�ances de cours. 


Pour cela, nous sommes demand�s � d�velopper une application avec langage C pour simuler le trafic urbain, une file de voiture avec 2 feux de circulation. Donc il nous faut �tudier � la fois :
\bigskip
\begin{itemize}
	\item Le mod�le math�matique du mouvement des voitures sur la route
 	\item L'algorithme pour visualiser le mod�le ci-dessus 
 	\item La r�alisation langage C de l'algorithme ci-dessus (Nous avons choisi le technique OpenGL pour le r�aliser)
\end{itemize}
\bigskip


Notre travail, ainsi que ce rapport, est organis� selon ces trois parties.


\chapter{IDM (Intelligent Driver Model)}
IDM est un mod�le de la simulation du trafic urbain. Il est d�velopp� par Treiber, Hennecke and Helbing en 2000 pour am�liorer les mod�les existants qui ont moins de caract�ristiques r�elles. 

\section{D�finition}
IDM d�crit une file de voitures qui se suivent, l'�tat du trafic en un moment pr�cis� est caract�ris� par la position, la vitesse et l'acc�l�ration des voitures dans la file. L'acc�l�ration d'une voiture, en effet, signifie l'action de conducteur. �a d�pend sa vitesse courante, la distance et la diff�rence de vitesse entre elle et la voiture juste de devant.


Les �quations d�crit ce mod�le.

\begin{subequations}
\begin{align}
%\label{equ:ch1_modele}
\frac{dv}{dt} & =a \left[1-\left(\frac{v}{v_{0}}\right)^{\delta}-
												\left(\frac{s^{*}}{s}\right)^{2}\right] \\
s^{*} & = s_{0}+vT+\frac{v\Delta v}{2\sqrt{ab}}
\end{align}
\end{subequations}

Les explications des param�tres initialis�s dans ces �quations sont les suivantes:
\begin{itemize}
	\item $v_{0}$ La vitesse d�sir�e sur une route libre.
	\item $T$ Le temps de s�curit� (temps minimum de r�action) quand la voiture suit une autre.
	\item $a$ Acc�l�ration g�n�rale. (selon l'habitude du conducteur)
	\item $b$ D�c�l�ration confortable g�n�rale.
	\item $s_{0}$ La distance minimum de s�curit� entre deux voitures, pare-choc � pare-choc.
	\item $\delta$ L'exposant de l'acc�l�ration qui est usuellement 4.
\end{itemize}
La variable $v$ et $s$ d�signent la vitesse courante et la distance de la voiture de devant. Et $s^*$ signifie une distance dynamique d�sir�e. 


On peut cons�d�rer la premi�re �quation en deux parties : une premi�re partie qui signifie l'acc�l�ration d�sir�e $\{a\left[1-\frac{v}{v_{0}}\right]^{\delta}\}$ 
et l'autre partie qui pr�sente la d�c�l�ration $\frac{s^{*}}{s}$ provoqu�e par la baisse de $s$. Alors, quand l'espace courant $s$ approche $s^*$, la d�c�l�ration provoqu�e va compenser l'acc�l�ration d�sir�e, par cons�quent, $\frac{dv}{dt}$ va diminuer. D'ailleurs, la d�c�l�ration va aussi augmenter quand la diff�rence de vitesse augmente fortement vers la voiture devant.



\chapter{Algorithme de l'application}
IDM est un mod�le qui est bas� sur l'�tat, c'est-�-dire quand on pr�cise un moment, on peut d�terminer les �tats (position, vitesse...) des voiture, et donc on peut afficher les voiture sur l'�cran. Ce que l'on a besoin, c'est seulement une boucle o� on mettre � jour les positions des voitures sur l'�cran. A partir de cette id�e, nous avons �tabli deux structures n�c�ssaires et l'algorithme principale.


Les structures de voiture et de la configuration sont d�finies ci-apr�s:
\begin{lstlisting}
nouvelle structure Config
{
    /** Param�tres pour l'affichage */
     windowWidth:entier;
    windowHeight:entier;
       roadWidth:r�el; 
        carWidth:r�el; 
       carLength:r�el; 
     
    /** (Longueur(m) virtuelle de la route/windowWidth(pixel). */
    roadVirtualLengthFactor:r�el; 
    
    /** Param�tres fonctionals */
    v0:r�el; /**La vitesse d�sir�e(km/h)*/    
     T:r�el; /**Temps de r�action*/
    s0:r�el; /**Gap minimal*/
     a:r�el; /**Acc�l�ration*/ 
     b:r�el; /**D�c�l�ration*/
};

nouvelle structure Car
{
    x:r�el; /**La postion courante*/
    v:r�el; /**La vitesse courante*/
    a:r�el; /**L'acc�l�ration*/
};
\end{lstlisting}

Et les algorithmes que nous avons cr��s sont d�crits au-dessous. Le premier est l'algorithme fondamental, qui d�crit le proc�dure d'ex�cution de l'application enti�re.
%Algo main
\begin{algorithm}[H]
\caption{Algorithme fondamental de l'affichage des voitures}
\label{algo:main}
\algsetup{indent=3em}
\begin{algorithmic}[1]
\REQUIRE { entr�es : config : Config initialis� du fichier de Configuration, \\
\makebox[40mm]{  }cars : un tableau de Car initialis�}
\ENSURE { sortie : Une file de voiture simul�e sur l'�cran }
%\STATE ini
\WHILE[La boucle qui dure jusqu'� la fin]{vrai} 
\STATE Dessiner l'arri�re plan
\FORALL{car dans le tableau Cars}
\STATE Dessiner la voiture sur l'�cran
\ENDFOR
\STATE Mettre � jour les param�tres des voitures pr�sent�s par cars
\ENDWHILE
\end{algorithmic}
\end{algorithm}


Algorithme ci-dessous est un sous-algorithme qui pr�sente le moyen de renouvellement des param�tres des voitures.
%Algo updateCars
\begin{algorithm}
\caption{Sous-algorithme pour renover les param�tres des voitures}
\label{algo:updatecars}
\algsetup{indent=3em}

\begin{algorithmic}
\REQUIRE { entr�es : cars : un tableau de Car}
\ENSURE { sortie : Le m�me tableau des voitures avec les param�tres renouvel�s.}
\FORALL{car dans le tableau Cars}
	\STATE Mettre � jour la position $x$ des voitures
	\STATE Mettre � jour la vitesse $v$ des voitures
	\STATE /* C'est la partie la plus importante, qui signifie l'action du conducteur intelligent et qui applique le mod�le IDM */
	\STATE Mettre � jour l'acc�l�ration $a$ des voitures
\ENDFOR

\end{algorithmic}
\end{algorithm}


L'algorithme suivant est la partie du coeur, qui applique le mod�le IDM. C'est un sous-algorithme plus d�taill� sur la m�thode pour mettre � jour le param�tre de l'acc�l�ration.
%Algo updateAcceleration
\begin{algorithm}[H]
\caption{Sous-algorithme pour mettre � jour le param�tre de l'acc�l�ration}
\label{algo:updateAcce}
\algsetup{indent=3em}
\begin{algorithmic}[1.1.1]
\REQUIRE { entr�es : cars : un tableau de $n$ Car,\\
\makebox[33mm]{  }config : Config initialis� du fichier de Configuration}
\ENSURE { sortie : Le m�me tableau des voitures avec les param�tres d'acc�l�ration renouvel�s.}
\FOR{i de $1$ � $n-1$}
	\STATE $car[i].a\leftarrow config.a
	\left[1-\left(\frac{car[i].v}{config.v_{0}}\right)^4-
		\left(
			\frac{config.s_{0}+car[i].v*config.T+\frac{car[i].v*(car[i].v-car[i+1].v)}{2\sqrt{config.a*config.b}}}{car[i+1].x-car[i].x}
		\right)^{2}
	\right]$
\ENDFOR

\end{algorithmic}
\end{algorithm}

\chapter{OpenGL}
Apr�s effectuer la premi�re partie de travail sur l'algorithme, nous passons � la suite pour la r�alisation de l'application. D'apr�s le but de ce projet, nous choisissons langage C pour la r�alisation. Cependant, ce que l'on veut faire, c'est de la programmation graphique, qui ne peut pas �tre r�alis�e en utilisant simplement ce que l'on a �tudi� pendant le cours. Alors nous avons fait de la recherche pour trouver une solution. Enfin nous avons choisi le technique OpenGL pour le faire.

\section{Introduction}
OpenGL(Open Graphics Library) est une librairie multiplate-forme qui d�finit un ensemble d'API pour faciliter la conception de l'application qui concerne les graphiques 3D/2D. Par rapport � DirectX(l'autre technique similaire sorti par Microsoft), OpenGL supporte plusieurs plate-formes. Par cons�quent elle est d�j� le standard de l'industrie. Elle contient environ 250 fonctions qui peuvent �tre utilis�es pour affichier des sc�nes trimentionnelles complex � partir des simples primitives g�om�triques.


Puisque OpenGL est simplement un ensemble des fonctions, elle seule ne nous permet pas de cr�er une fen�tre pour montrer notre application. Donc il existe plusieus extensions d'OpenGL qui fournit d'autres fonctionnalit�s. GLUT(OpenGL Utility Toolkit) est un bon choix.


\section{Environnement de d�veloppement}
Pour r�aliser cette application, nous avons choisi CodeBlocks comme notre outil de d�veloppement. Pour �tablir l'environnement de OpenGL, nous avons suivi les d�marches suivantes:

\begin{enumerate}
	\item T�l�charger et installer CodeBlocks.
	\item T�l�charger les fichiers de GLUT.
	\item Placer glut32.dll � \verb�c:\windows\system�; glut32.lib � \verb�c:\program files\mingw\lib�, et
glut.h � \verb�c:\program files\mingw\include\GL�. Ce sont tous les positions par d�fault, en principe, les fichiers *.dll doit toujours �tre plac� sous le r�pertoire syst�me Windows; les *.lib et *.h sont mis respectivement sous les r�pertoire de librairie et de fichier d'en-t�te du compilateur.
	\item Dans CodeBlocks, cr�er un nouveau projet GLUT et poursuivre les d�marches.Figure\ref{fig:ch3_newglut}
	\item Enfin on obtient un projet d'exemple, mais on doit quand m�me ajouter quelques choses pour le faire marcher. Ajouter \verb%#include<windows.h>% au d�but du fichier Main.
	\item Maintenant on peut d�j� lancer cet application d'exemple. Figure\ref{fig:ch3_glexemple}
\end{enumerate}	

\begin{figure}[!ht]
	\centering
		\includegraphics[width=14cm]{pics/ch3_newglut.png}
	\caption{Nouveau projet GLUT}
	\label{fig:ch3_newglut}
	\end{figure}
	
	\begin{figure}[!ht]
		\centering
			\includegraphics[width=14cm]{pics/ch3_glexemple.png}
		\caption{Exemple projet GLUT}
		\label{fig:ch3_glexemple}
	\end{figure}
	
\section{Structure d'une application OpenGL}
\label{ch3_structureopengl}
Pour ma�triser OpenGL, beaucoup de conna�ssances sont n�cessaires, mais ici, on peut quand m�me montrer une structure d'une application simplifi�e OpenGL.


La fonction main est g�n�ralement comme �a:
\begin{lstlisting}
int main(int argc, char** argv)
{
    glutInit(&argc,argv);
    glutInitDisplayMode(GLUT_DOUBLE| GLUT_RGB);
    glutInitWindowSize(config.windowWidth,config.windowHeight);
    glutInitWindowPosition(0,0);
    glutCreateWindow(''Traffic Simulation'');
    glClearColor(25.0/255,134.0/255,19.0/255,0.3);//Couleur de l'arri�re plan
    glShadeModel(GL_SMOOTH);
    /*Function callback. Fonction myReshape sera invoqu� 
     *quand la fen�tre change sa forme.*/
    glutReshapeFunc(myReshape);
    /*Fonction myDisplay actualiser l'�cran*/
    glutDisplayFunc(myDisplay);
    /*mouse est invoqu� quand un bouton de la souris est appuy�*/
    glutMouseFunc(mouse);

    /*Cr�er un menu*/
    glutCreateMenu(menuFonc);
    glutAddMenuEntry(''Start'',MENU_START);
    glutAddMenuEntry(''Renew configuration'',MENU_RENEW);
    glutAddMenuEntry(''Synchronize the traffic lights'',MENU_SYNC);
    glutAttachMenu(GLUT_RIGHT_BUTTON);//Lier ce menu avec un bouton
    
    /*Cette derni�re ligne commence la boucle principale de cet application*/
    glutMainLoop();
}
\end{lstlisting}

\chapter{Mise en oeuvre}
%mise en oeuvre
Sur le site\footnote{\url{http://www.xdevelop.at}}, il y a des tutoriaux assez d�taill�s pour commence � programmer pour AndroPOD. Ils offrent m�me un petit projet d'exemple qui a une fonctionnalite �l�mentaire pour communiquer entre l'application et AndroPOD par la connexion TCP. Ayant auparavant des connaissances sur Java et la programmation des r�seaux, nous avons �tudi� ce petit programme qui a facilit� notre travail.


Notre application poss�de une interface ci-dessous \ref{fig:ch4_interface}:

\begin{figure}[!ht]
\centering
\includegraphics[width=6cm]{pics/ch4_interface.png}%
\caption{L'interface de l'application}%
\label{fig:ch4_interface}%
\end{figure}
%\\[\intextsep]
%\begin{minipage}{\textwidth}
%\centering
%\includegraphics[width=5cm]{pics/ch4_interface.png}
%\label{fig:ch4_interface}
%\end{minipage}
%\\[\intextsep]

A la fin, nous avons reli� toutes les trois parties --- mobile, AndroPOD et ordinateur, et nous avons r�ussit de contr�ler le robot avec notre mobile.

\begin{figure}[!ht]
\centering
\includegraphics[width=14cm]{pics/ch4_ensemble.JPG}%
\caption{L'oeuvre finale}%
\label{fig:ch4_ensemble}%
\end{figure}

\chapter{Conclusion}
Notre projet contient trois parties principales : le syst�me Android, la carte AndroPOD et le petit robot. Nous avons fait beaucoup de recherche et d'�tude pour connaitre chacune partie, et � la fin de notre projet, nous avons r�ussi � contr�ler le robot avec notre mobile Android, ayant AndroPOD comme le pont entre eux, et une application install�e sur le mobile.


La programmation sous Android est en vogue pour le moment. Ce projet nous permet pas seulement d'avoir une connaissance fondamentale du d�veloppement sous Android, mais encore d'avoir une impression sur le contr�le de mat�riel. C'est aussi int�ressant de faire un lien entre notre mobile Android et un petit robot. 
A la fin, nous voudrions remercier notre encadrant M. Pascal Makris, qui nous a accompagn� tout au long du d�veloppement du projet, en nous guidant par les r�unions fr�quentes et en nous offrant les mat�riels requis.


\annexes

\end{document}

